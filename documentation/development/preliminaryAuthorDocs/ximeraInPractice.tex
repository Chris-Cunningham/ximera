\documentclass{amsart}
\usepackage{multicol}
\usepackage{kmath,kerkis}
\usepackage{multirow}
\usepackage{fancyvrb}
\usepackage{xcolor}

\newcommand\code[1]{{\bfseries\texttt{#1}}}


\fvset{commandchars=+(),formatcom=\color{blue!50!black}}

\begin{document}
\title{XIMERA: Markup in Practice}
\maketitle

The goal of the XIMERA Project is to allow authors to write online
interactive content though \LaTeX\ documents.


\section{Writing basic questions}


The XIMERA Project offers four basic problem-like environments:
\code{exercise}, \code{problem}, \code{question}, and
\code{hint}. From the interactive viewpoint, each of these
environments does basically the same thing, providing a question for
the student.

An \code{exercise} is for checking computational or rote performance:

\begin{Verbatim}
  +color(red)\begin{exercise}
    An exercise.
  +color(red)\end{exercise}
\end{Verbatim}

A \code{problem} is for a more challenging question.

\begin{Verbatim}
    +color(red)\begin{problem}
    A problem.
    +color(red)\end{problem}
\end{Verbatim}

A \code{question} is for more open-ended exploration. 

\begin{Verbatim}
    +color(red)\begin{question}
    A question.
    +color(red)\end{question}
\end{Verbatim}

A \code{hint} is a sub-question that will help with solving the main
question.



\section{Adding solutions and answer fields}


As coded above, none of the problem-like environments have answers,
they are simply displayed questions.  For a problem-like environment to
have a solution, we must add a \code{solution} environment.


\begin{Verbatim}
  \begin{problem}
    A problem.
    +color(red)\begin{solution}
      A solution. 
    +color(red)\end{solution}
  \end{problem}
\end{Verbatim}

However, as it stands, the student will be presented with a problem,
and then be able to ``click'' to see the solution. To have an answer
field we to add the \code{answer}.

\begin{Verbatim}
  \begin{problem}
    A problem.
    \begin{solution}
      A solution. 
      +color(red)\answer{The answer.}
    \end{solution}
  \end{problem}
\end{Verbatim}

The default type for an answer is a mathematical expression. However,
there are other choices: \code{free-response}, \code{image}, and
\code{multiple-choice}. The option \code{free-response} provides a
text field that is ungraded. The option \code{image} allows the
student to upload an image as the solution. The option
\code{multiple-choice} accepts a pre-defined (non-case sensitive)
answer.

\begin{Verbatim}
  \begin{problem}
    A multiple choice problem.
    \begin{itemize}
      \item Choice a. % with label a
      \item Choice b. % with label b
    \end{itemize}
    \begin{solution}
      A solution. 
      +color(red)\answer[multiple-choice]{a}
    \end{solution}
  \end{problem}
\end{Verbatim}



\section{Questions with multiple-parts and hints}

To add questions with multiple parts, simply add more answers to the
solution environment.

\begin{Verbatim}
  \begin{problem}
    A problem. 
    +color(red)\begin{solution}
      First solution. 
      \answer{First answer.}
    +color(red)\end{solution}
    A follow-up problem.
    +color(red)\begin{solution}
      Second solution. 
      \answer{Second answer.}
    +color(red)\end{solution}
  \end{problem}
\end{Verbatim}


To add hints to the question, add a \code{hint} within the
\code{solution} environment.



\begin{Verbatim}
  \begin{problem}
    A problem. 
    \begin{solution}
      +color(red)\begin{hint}
        A hint.
      +color(red)\end{hint}
      A solution. 
      \answer{First answer.}
    \end{solution}
  \end{problem}
\end{Verbatim}

To make the hints more Socratic, they themselves can be questions
with solutions with/or without answer fields:

\begin{Verbatim}
  \begin{problem}
    A problem. 
    \begin{solution}
      +color(red)\begin{hint}
        A hint.
        +color(red)\begin{solution}
        The solution to the question asked by the hint.
        +color(red)\end{solution}
      +color(red)\end{hint}
      A solution. 
      \answer{First answer.}
    \end{solution}
  \end{problem}
\end{Verbatim}



\section{Presenting one question of a theme of questions}

To allow the student to master a concept, it is often useful to have a
group of questions that are all presented as a single question in the
online experience. To do this, use the \code{shuffle} command.


\begin{Verbatim}
  +color(red)\begin{shuffle}
    \begin{problem}
      A variation of a problem, solution etc.
    \end{problem}
    
    \begin{problem}
      Another variation of a problem, solution etc.
    \end{problem}
  +color(red)\end{shuffle}
\end{Verbatim}

By default \code{shuffle} simply presents a variation of the problem
at random for the student to solve with the option of repeating the
question, though due to the randomization, the student will probably
be asked to solve a different variation of the question. The highest
score achieved on any of the variations of the question is recorded as
the students performance.

There are several options for \code{shuffle}, 
\begin{description}
\item[\code{once}] Chooses a problem variation at random and allows
  exactly one attempt to solve the problem.
\item[\code{order}] Presents the problems in the order listed to the student, should the student choose to attempt the problem multiple times. 
\item[\code{mastery}] Presents the problems at random and encourages
  the student to attempt the problem multiple times. Here the score is
  based on the aggregate performance, rather than the single highest
  attempt.
\end{description}


\begin{Verbatim}
  +color(red)\begin{shuffle}[mastery]
    \begin{exercise}
      A variation of an exercise, solution etc.
    \end{exercise}
    
    \begin{exercise}
      Another variation of an exercise, solution etc.
    \end{exercise}
  +color(red)\end{shuffle}
\end{Verbatim}


In addition there is the global option \code{adaptive}. If this option
is selected, the problems will no longer be random. Instead, the
adaptive learning algorithm will select a mid difficulty variation of
the problem. If the student is successful, a harder variation will be
given next. If the student is unsuccessful, then the student will be given an easier question for the next attempt. 
\end{document}
